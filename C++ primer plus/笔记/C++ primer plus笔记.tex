\documentclass[a4paper,12pt]{article}
\usepackage{setspace}
\usepackage{indentfirst}
\usepackage{graphicx}
\usepackage[UTF8]{ctex}
\usepackage{amsmath} 
\usepackage[dvipsnames]{xcolor}
\usepackage{listings}
\usepackage{indentfirst}
\usepackage{xcolor}
\lstset { %
    language=C++,
    backgroundcolor=\color{black!5}, % set backgroundcolor
    basicstyle=\footnotesize,% basic font setting
}
\begin{document}
\title{C++ 学习笔记}
\date{\today}
\maketitle
\begin{spacing}{2}
\pagenumbering{arabic}
\tableofcontents
\newpage
\pagenumbering{arabic}
\end{spacing}
\begin{spacing}{2}
\section{指针,数组,结构体}
\subsubsection{在输入字符的时候有三种情况}:
\begin{itemize}
\item 当输入两次字符串并且需要换行的时候,可以用cin.getline()或者cin.get()两种方法代替
\begin{lstlisting}
cout<<"Enter your name:";
cin>>name;
cout<<"Enter name of dessert:";
cin>>dessert;
\end{lstlisting}
假如是以上情况,那么未能等到输入dessert的名字的时候,就结束了。这是因为cin通过使用空白(空格,制表符和换行符)来确定字符串的结束位置,这意
味着cin在获取字符数组输入时只读取一个字符。
\item 所以可以使用cin.getline(变量名,数组长度)。或者cin.get(变量名,数组长度)。但在使用cin.get()也需要注意的是,当使用第一次cin.get()之后,在第一次调用换行符还在输入列中,因此第二次调用cin.get()的时候看到的第一个字符便是换行符,因此get()认为已经到了结尾,而没发现任何可以读取的内容。
\begin{lstlisting}
cin.get(name,size);
cin.get();
cin.get(dessert,size);
\end{lstlisting}
\item 在输入数字之后,由于回车键生成的换行符留在了输入列中,cin.getline()看到换行符之后,将认为是一个空行,想继续输入字符的话,可以调用cin.get()方法
\begin{lstlisting}
int year;
cin>>year;
cin.get(name,size);

(cin>>year).get();
cin.getline(name,size);
\end{lstlisting}
\end{itemize}

\subsubsection{自动存储、静态存和动态存储}
\begin{itemize}
\item\textbf{自动存储}当函数内部定义的常规变量使用自动存储空间,被称为自动变量,这意味着它们在所属的函数被调用时自动产生,在该函数结束时消亡。自动变量通常存储在\textbf{栈}中。这意味着执行代码块时,其中的变量将依次加入到栈中,而在离开代码块时,将按照相反的顺序释放这些变量,这被称为后进先出。因此,在程序执行过程中,栈将不断放大缩小
\item\textbf{静态存储}静态存储是整个程序执行期间都存在的存储方式。使变量成为静态的方式有两种:一种是在函数外面;另一种实在变量时使用关键词:static:
\begin{lstlisting}
double increase()
{
	static double fee = 56.5;
	fee = fee + 1;
	return fee;
		
}
int main()

{
	double  a=increase();
	cout << a << endl;
	float b=increase();
	cout << b;
	return 0;

}
\end{lstlisting}a=57.5,b=58.5因为在每一次调用fee本身都会增加一,并不会随着increase()结束调用而消亡。
\item \textbf{动态存储} new 和delete 运算符提供了一种比自动变量和静态变量更灵活的方法。它们管理了一个内存池吗,在C++中被称为自由存储空间(free store)或 堆(heap)。该内存池用于静态变量和自动变量的内存是分开的。
\item \textbf{**的用法}
\begin{lstlisting}
#include<iostream>
#include <stdio.h>
#include<string>
#include<cctype>
using namespace std;
struct antarctica_years_end
{
	int year;
};
int main()

{
	antarctica_years_end  s01, s02, s03;
	s01.year = 1998;
	antarctica_years_end* pa = &s02;
	pa->year = 1999;
	antarctica_years_end trio[3];
	trio[0].year = 2003;
	cout << trio->year << endl;
	const  antarctica_years_end* arp[3] = { &s01,&s02,&s03 };
	cout << arp[1]->year << endl;
	const antarctica_years_end** ppa = arp;
	auto ppb = arp;
	cout << (*ppb)->year << endl;
	cout << (*(ppb+1))->year << endl;
	return 0;

}
\end{lstlisting}
\begin{figure}[h]
\includegraphics[scale=1]{Figure 1.PNG}
\caption{**基本输出}
\label{figure 1}
\end{figure}
\end{itemize}
\end{spacing}
\end{document}